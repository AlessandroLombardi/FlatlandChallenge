%----------------------------------------------------------------------------------------
%	PACKAGES AND DOCUMENT CONFIGURATIONS
%----------------------------------------------------------------------------------------

% !TeX spellcheck = it
\documentclass[11pt, a4paper, hidelinks]{article}

\usepackage{anyfontsize}
\usepackage{siunitx} % Provides the \SI{}{} and \si{} command for typesetting SI units
\usepackage{graphicx} % Required for the inclusion of images
\usepackage{subcaption}
%\usepackage{natbib} % Required to change bibliography style to APA
\usepackage{amsmath} % Required for some math elements
\usepackage[export]{adjustbox}
\usepackage{eurosym} % euro simbol
\usepackage{hyperref} % hyperlink
\usepackage[utf8]{inputenc}
\usepackage{bookmark}
\usepackage{float}
\usepackage{epigraph}
\usepackage{quoting}
\usepackage{newlfont}
\usepackage{color}
\usepackage{geometry}
\usepackage{algorithm2e}
% To add numbers in algorithm lines
\LinesNumbered

\renewcommand{\labelenumi}{\alph{enumi}.} % Make numbering in the enumerate environment by letter rather than number (e.g. section 6)

\begin{document}
\begin{titlepage}

\begin{center}
{{\Large{\textsc{Alma Mater Studiorum $\cdot$ Universit\`a di Bologna}}}}
\rule[0.1cm]{15.8cm}{0.25mm}
\\\vspace{3mm}
%
%
{\Large{Dipartimento di Informatica - Scienza e Ingegneria\\
Artificial Intelligence}}


\end{center}

\vspace{20mm}

\begin{center}{
%
%
	{\LARGE{\textbf{Flatland Challenge}}}}
\end{center}

\vspace{15mm}

{\begin{center}
	 \large{Project Presentation}
\end{center}}

\vspace{32mm} \par \noindent

\begin{minipage}[t]{0.47\textwidth}
%
%
{\large{ Professor \vspace{2mm}\\{\textbf{Andrea Asperti}
}\\\\\\}}
\end{minipage}
%
\hfill
%
\begin{minipage}[t]{0.47\textwidth}\raggedleft{}{
{\large{ Students
\vspace{2mm}\\
\textbf{Alessandro Lombardi\\Fiorenzo Parascandolo} }}}
\end{minipage}

\vspace{31mm}

\begin{center}
Academic Year {2019/2020}
\end{center}

\end{titlepage}

{\tableofcontents}
\thispagestyle{empty}

\newpage
\setcounter{page}{1}

\section{Flatland}\label{sec:flatland}

The analyzed version is the 2.1.10

\subsection{The classes RailAgentStatus and EnvAgent}\label{subsec:the-classes-railagentstatus-and-envagent}

RailAgentStatus extends Python IntEnum and assumes the following values:
\begin{itemize}
	\item READY\_TO\_DEPART (0) the agent is not in the grid yet (position is None), the prediction is to stay at the starting position.
If a MOVE\_* action is performed during this state it becomes ACTIVE\@.
	\item ACTIVE (1) the agent is in the grid (position is not None) and hasn't reached the target yet, the prediction is the remaining path.
	\item DONE (2) the agent is still in the grid (position is not None) but has already reached the target, the prediction is to stay at the target forever.
	\item DONE\_REMOVED (3) the agent has reached the target and it's removed from the grid.
\end{itemize}

Grid4TransitionsEnum extends Python IntEnum and assumes the following values: NORTH (0), EAST (1), SOUTH (2), WEST (3).
Grid4TransitionsEnum is used to indicate absolute directions, related to the environment, like a compass.
Possible usages are storing where the agent is facing or computing legal actions, for example including as observation a one hot encoding of the directions where the agent can move.
\\
EnvAgent class models the agent and encapsulates in its internal state the following attributes:
\begin{itemize}
	\item initial\_position: Tuple[int, int], initial coordinate.
	\item initial\_direction: Grid4TransitionsEnum, the initial agent facing direction.
	\item direction: Grid4TransitionsEnum, the current facing direction.
	\item target: Tuple[int, int], the final coordinate.
	\item moving: bool, True if the agent is in a moving state.
	\item speed\_data: dictionary, TODO
	\item malfunction\_data: dictionary, TODO
	\item status: RailAgentStatus, the current agent status.
\end{itemize}

The speed of an agent contains the keys 'position\_fraction' used as a counter of the percentage of completion of an movement from a cell to another, 'speed' the value between 0 and 1 used to increment the 'position\_fraction' and 'transition\_action\_on\_cellexit' which contains the action to perform on the next cell (if it completes the one in the current step).

The malfunction of an agent contains the keys 'malfunction' which contains how many steps are necessary to fix the agent, 'malfunction\_rate', the mean rate (average number of events in an interval) of the Poisson distribution, 'next\_malfunction' the number of steps the next malfunction will occur and 'nr\_malfunctions' the number of previous malfunctions.

\subsection{The class RailEnv}\label{subsec:the-class-railenv}

From the documentation
\begin{quotation}
	RailEnv is an environment inspired by a (simplified version of) a rail
    network, in which agents (trains) have to navigate to their target
    locations in the shortest time possible, while at the same time cooperating
    to avoid bottlenecks.
\end{quotation}

In the \textit{step} function the number of steps is updated and if the overall task is still uncompleted, for each agent the associated reward is initially put to zero, a malfunction is tried to be induced and the specific step is performed.
The info of the agent are prepared and finally the end malfunctions are repaired.
Agents are handled in the order in which are passed.

\subsubsection*{Environment Actions}
The avaiable actions are:
\begin{itemize}
	\item DO\_NOTHING (0) Default action if None has been provided or the value is not within this list.
If agent.moving is True then the agent will MOVE\_FORWARD\@.
    \item MOVE\_LEFT (1) If agent.moving is False then becomes True.
If it's possible turn the agent left, changing its direction, otherwise if agent.moving is True tries the action MOVE\_FORWARD\@.
    \item MOVE\_FORWARD (2) If agent.moving is False then becomes True.
It updates the direction of the agent and if the new cell is a dead-end the new direction is the opposite of the current.
    \item MOVE\_RIGHT (3) If agent.moving is False then becomes True.
If it's possible turn the agent right, changing its direction, otherwise if agent.moving is True tries the action MOVE\_FORWARD\@.
    \item STOP\_MOVING (4) If agent.moving is True then becomes False.
A penalty will be added.
Stop the agent in the current occupied cell.
\end{itemize}

\begin{algorithm}[H]
	\uIf{agent is in DONE or in DONE\_REMOVED ($1^{\text{th}}$ case)}{
		no reward is computed\\
		\Return{}
	}
	\uIf{agent is in READY\_TO\_DEPART ($2^{\text{th}}$ case)}{
		if the provided action is a MOVE\_* type and the initial cell is free the agent become ACTIVE and is initialized\\
		reward is computed\\
		\Return{}
	}
	\uIf{agent is in malfunction (($3^{\text{th}}$ case))}{
		reward is computed\\
		\Return{}
	}
	\uIf{agent is at the beginning of a cell}{
		update agent.moving considering the observations above depending on the different action types.\\
		\uIf{agent.moving}{
			the wanted action validity is first checked and if it is valid (considering also the possibility to backup from an invalid MOVE\_RIGHT or MOVE\_LEFT to a valid MOVE\_FORWARD) action is stored otherwise agent.moving becomes False and penalties are added, in this process agent.moving and agent.speed\_data['transition\_action\_on\_cellexit'] are updated
		}
	}

	\eIf{agent.moving ($4^{\text{th}}$ case)}{
		Updates the percentage of completion then if it is completely arrived on the next cell, before updating the position, the direction and clears the completion percentage it checks whether the new cell is free. Until the cell remains occupied in the future executions the agent will repeat this process.\\
		reward is computed
	}{
		reward is computed
	}
 	\caption{The \textit{\_step\_agent} algorithm}\label{alg:flatland_agent_loop}
\end{algorithm}
\noindent
\\
\\
Some useful questions:
\begin{itemize}
	\item An agent can stop during an action between two cells?
Absolutely no, it's like any other action.
	\item Requested actions during a malfunction are ignored?
Yes.
	\item Requested actions during a not completed movement are saved for after execution?
No, because the condition at line 11 is not executed and the conditions in line 16 check whether the cell is free and possibly complete agent data.
Actions are not allowed to change within the cell, each agent can only chose an action to be taken when entering a cell.
This action is then executed when a step to the next cell is valid.
	\item How is possible to understand if an agent is ready to perform an action?
The entry info\_dict["action\_required"] returned by the function \textit{step} of \textbf{RailEnv} contains True for the given agent.
This doesn't mean that the action will be successfully executed due to the presence of malfunctions or blocking agents, in this case info\_dict["action\_required"] will remain True.
	\item When an agent reaches DONE is removed automatically the following step?
TODO Yes.
	\item Does an agent automatically pass from READY\_TO\_DEPART to ACTIVE at the beginning?
TODO No. A MOVE\_* is necessary.
	\item Do collisions occur?
No, agents check if the cell is free before moving.
Deadlocks are possible when two agents are one in front the other without any chance to change path.
\end{itemize}

\subsubsection*{Malfunctions}
The strategy depends on the passed \textit{malfunction\_generator\_and\_process\_data}.
TODO

\begin{itemize}
	\item A malfunction can occur during the resolution of another?
TODO
	\item An agent could have a malfunction during the completion of an action between two cells?
TODO Yes.
\end{itemize}

\subsubsection*{Speed}
The different speed profiles (speed is between 0 and 1) can be generated using the schedule\_generator.
Speed configurations can be build using \textbf{ScheduleGenerator}s.
TODO

\subsubsection*{Rewards}
The rewards are based on the following values:
\begin{itemize}
	\item invalid\_action\_penalty which is currently set to 0, penalty for requesting an invalid action
	\item \textbf{step\_penalty} which is -1 * alpha, penalty for a time step.
	\item \textbf{global\_reward} which is 1 * beta, a sort of default penalty.
	\item stop\_penalty which is currently set to 0, penalty for stopping a moving agent
	\item start\_penalty which is currently set to 0, penalty for starting a stopped agent
\end{itemize}

The full step penalty is computed as the product between step\_penalty and agent.speed\_data['speed'].
There are different rewards for different situations:

\begin{itemize}
	\item single agents that are in DONE or in DONE\_REMOVED have zero reward ($1^{\text{th}}$ \textit{\_step\_agent} case).
	\item all agents that have finished in this episode (checked at the end of the \textit{step}) or previously (checked at the beginning of the \textit{step}), have reward equal to the global\_reward (when in \textit{step} all agents have reached their target)
	\item full step penalty is assigned when an agent is READY\_TO\_DEPART and in the current turn moves or stay there ($2^{\text{th}}$ \textit{\_step\_agent} case), or when is in malfunction ($3^{\text{th}}$ \textit{\_step\_agent} case).
	\item full step penalty plus the other penalties (invalid\_action\_penalty, stop\_penalty and start\_penalty) when the agent is finishing actions or start new ones ($4^{\text{th}}$ \textit{\_step\_agent} case).
Currently the other penalties are all set to zero.
\end{itemize}

So each train starts counting rewards since the beginning, not since it becomes ACTIVE\@.
Currently it is possible to say that rewards are always full step excluding the end of the episode and the single agents that have finished which have reward equal to 0.

\section{Multi Agent Reinforcement Learning (MARL)}\label{sec:multi-agent-reinforcement-learning}

This section provides an overview of some useful works and theory behind MARL that we consider useful to suggest approaches or solutions to the Flatland Challenge.

\subsection{Theoretical background}\label{subsec:marl-theoretical-background}

\subsubsection{Introduction}

\begin{quoting}[font=itshape, begintext={"}, endtext={"\cite{MARL_definition}}]
Specifically, MARL addresses the sequential decision-making problem of multiple autonomous agents that operate in a common environment, each of which aims to optimize its own long-term return by interacting with the environment and other agents.
\end{quoting}

MARL algorithms can be divided into three groups~\cite{zhang2019multiagent}:

\begin{itemize}
	\item \textbf{Fully cooperative}, where agents collaborate to optimize a common long-term return.
	\item \textbf{Fully competitive}, where the return of agents usually sum up to zero.
	\item \textbf{Mix of the two}, where both cooperative and competitive agents are involved.
\end{itemize}

We consider the Flatland Challenge a fully cooperative environment since each agent apparently compete to reach faster its destination and gain rails portions but the problem must consider the common time minimization goal.
There exist two closely related theoretical frameworks for MARL~\cite{zhang2019multiagent}:

\begin{itemize}
	\item \textbf{Markov/Stochastic Game}.
All agents share the same state and differently from classical single agent's \textbf{Markov Decision Process (MDP)}, the optimal performance of each agent is controlled not only by its own policy, but also the choices of all other players of the game.
Usually agents share a common reward function but it is possible to have different functions like in the team-average reward setting.
	\item \textbf{Decentralized POMDP (Dec-POMDP)}.
As a MDP can be extended to a \textbf{Partially Observable Markov Decision Process (POMDP)} when information that can be accessed at a given state is incomplete, similarly a Markov Game can be extended to a Dec-POMDP\@.
In multi-agent scenario, an agent may not only depend on the information it has autonomously gathered, it will also be influenced by the choices of other agents, which are partially observable.
In Dec-POMDP each agent has its own local observation of the system state, that without other agents' observations, leads to the impossibility to maintain a global belief state.
To overcome this problem, agents can exploit levels of coordination among them to obtain the full observability of a state by combining the individual observations from each member~\cite{castaneda}.
Dec-POMDP approaches are usually considered more difficult to solve than others, especially when the number of agents is greater than two.
	\item \textbf{Extensive-Form Game} inspired from computational game theory, it handles imperfect information.
It is usually used in mixed or competitive environments.
\end{itemize}

In the Flatland environment a state is surely represented by the positions and orientations of each agent in the map, since the railroad is static and agents influence it only by moving and interrupting paths.
Flatland environment allows to personalize how agents perceive the world implementing custom observations.

The involvement of Deep Learning to tackle the problem of MARL defines a new specific subject called \textbf{Multi-agent Deep Reinforcement Learning (MADRL)}.
~\cite{Hernandez_Leal_2019} presents four categories of recent MADRL works:
\begin{itemize}
	\item \textbf{Analysis of emergent behaviors}, in general, they do not propose learning algorithms, their main focus is to analyze and evaluate DRL algorithms in a multi-agent environment.
	\item \textbf{Learning communication}, they study communications techniques to share information.
	\item \textbf{Learning cooperation}, they directly explore approaches based on actions and observations to build multi-agent systems.
	\item \textbf{Agents modeling agents}, they study how agents reason about others to fulfill a task.
\end{itemize}

\subsubsection{Challenges}

MARL frameworks inevitably adds many challenging problems on the single-agent scenario.
\begin{quoting}[font=itshape, begintext={"}, endtext={"\cite{Hernandez_Leal_2019}}]
Learning in multiagent settings is fundamentally more difficult than the single-agent case due to the presence of multiagent pathologies, e.g., the moving target problem (non-stationarity) [2, 5, 10], curse of dimensionality [2, 5], multiagent credit assignment [31, 32], global exploration [8], and relative overgeneralization [33, 34, 35].
\end{quoting}

Below follow some problems that may affect the Flatland Challenge.

\paragraph{Non-Stationarity}~\newline

\begin{quoting}[font=itshape, begintext={"}, endtext={"\cite{papoudakis2019dealing}}]
In Markov games, the state transition function $T$ and the reward function of each agent $r_i$ depend on the actions of all agents.
During the training of multiple agents, the policy of each agent changes through time.
As a result, each agents’ perceived transition and reward functions change as well.
Single-agent RL procedures which commonly assume stationarity of these functions might not quickly adapt to such changes.
\end{quoting}

In a single-agent environment, an agent is concerning only the outcome of its own actions.
In a multi-agent scenario, an agent observes not only the outcomes of its own action but also the behavior of other agents.
Agents may interact with each other and learn concurrently leading to a continuous reshape of the environment and to non-stationarity~\cite{zhang2019multiagent}.
For example the classical DQN does not provide working solutions, some derivations have been proposed to deal with this problem such as \textbf{Deep Repeated Update Q-network (DRUQN)}~\cite{castaneda}, \textbf{Deep Loosely Coupled Q-network (DLCQN)}~\cite{castaneda} and \textbf{ multi-agent concurrent DQN}.
Other techniques to adapt classical experience replay to multi-agent environment have been proposed such as \textbf{Hysteretic-DQN (HDQN)} and \textbf{Lenient-DQN (LDQN)}~\cite{Nguyen_2020}.

There are different ways to tackle the non-stationary problem, as illustrated by~\cite{papoudakis2019dealing}.
\begin{itemize}
	\item \textbf{Centralized Critic Architecture} based on an actor-critic algorithm.
The critics' training is centralized and has access to the observations and actions of all agents, while the actors' training is decentralized.
An example is the \textbf{Multi-Agent Deep Deterministic Policy Gradient (MADDPG)} algorithm.
In MADDPG each agent uses a centralized critic and a decentralized actor.
Since the training of each agent depends on the observations and actions of all the other agents, each agent perceives the environment as stationary.
	\item \textbf{Decentralized Learning} Techniques using self-play.
	\item Opponent Modelling.
	\item \textbf{Meta-Learning}.
	\item Communication.
Either accessing hidden layers as \textbf{CommNet} or feeding other agents' neural networks as \textbf{Reinforced Inter-Agent Learning}.
\end{itemize}



\paragraph{Partial observability}~\newline~\newline
As mentioned, in the single-agent scenario this type of problem is usually modelled with a POMDP\@.
\textbf{Deep Recurrent Q-Networks (DRQN)} proposed using recurrent neural networks, in particular, Long Short-Term Memory (LSTMs) cells in DQN, to introduce a memory capability.
An extension of DRQN for multi-agent environments is \textbf{Deep Distributed Recurrent Q-network (DDRQN)}~\cite{Nguyen_2020}.
Another technique to deal with partial observability is \textbf{Deep Recurrent Policy Inference Q-network (DRPIQN)} learned by adapting network’s attention to policy features and their own Q-values at various stages of the training process.
Methods to address partial observability in Markov Games usually involve communication (\textbf{Reinforced Inter-Agent Learning (RIAL)} and \textbf{Differentiable Inter-Agent Learning (DIAL)}) or parameter sharing (\textbf{PS-DQN}, \textbf{PS-DDPG}, \textbf{PS-A3C} and \textbf{PS-TRPO}).
Alternatively to the Markov Game, the Dec-POMDP can be used to model this type of scenario in a more classical \textbf{centralized learning for decentralized execution} fashion or in a more modern \textbf{decentralized learning for decentralized execution} way.


\paragraph{Scalability}~\newline~\newline
To handle non-stationarity, each individual agent may need to account for the joint action space, whose dimension increases exponentially with the number of agents, this is  also referred to as the combinatorial nature of MARL\@~\cite{zhang2019multiagent}.
Many methods have been proposed to tackle this problem, one of them is the extension of \textbf{Curriculum Learning} for a multi-agent scenario.


\paragraph{Information Structures and Training Schemes}~\newline~\newline
In the single-agent case is easier to understand what information is visible to the agent.
In Markov games is sufficient to observe the current state, while on extensive-form games agents may need to recall the history of past decisions.
In addition agents struggle to fully access information like rewards and policies of other agents, increasing the non-stationarity viewed by by individual agents~\cite{Nguyen_2020}~\cite{zhang2019multiagent}.
These considerations led to the development of different training schemes such as centralized learning for decentralized execution, which originated from the works on the Dec-POMDP setting and has been widely adopted in recent MADRL works, and \textbf{fully decentralized}.
The former has become a standard as simulators are usually involved in MARL training, there are different types of communications between agents and the central controller such as centralized learning, concurrent learning and parameter sharing\cite{Nguyen_2020}.
Usually decentralized settings may allow some sort of communication to address the non-convergence issue typical of the independent learning, this strategy is also referred as \textbf{decentralized setting with networked agents}.


\paragraph{Multi-agent credit assignment}

\subsubsection{Algorithms}


\newpage
~\nocite{*}
\bibliography{bibliography}
\bibliographystyle{plain}

\end{document}