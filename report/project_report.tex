%----------------------------------------------------------------------------------------
%	PACKAGES AND DOCUMENT CONFIGURATIONS
%----------------------------------------------------------------------------------------

% !TeX spellcheck = it
\documentclass[11pt, a4paper, hidelinks]{report}

\usepackage{anyfontsize}
%\usepackage{siunitx} % Provides the \SI{}{} and \si{} command for typesetting SI units
\usepackage{graphicx} % Required for the inclusion of images
\usepackage{subcaption}
\usepackage[round]{natbib} % Required to change bibliography style to APA
\usepackage{amsmath} % Required for some math elements
\usepackage[export]{adjustbox}
\usepackage{eurosym} % euro simbol
\usepackage{hyperref} % hyperlink
\usepackage[utf8]{inputenc}
\usepackage{bookmark}
\usepackage{float}
\usepackage{epigraph}
\usepackage{quoting}
\usepackage{newlfont}
\usepackage{color}
\usepackage[a4paper, total={6.2in, 8in}]{geometry}
\usepackage{algorithm2e}
\usepackage{titletoc}
\usepackage{listings}
\usepackage{algorithmic}

\bibliographystyle{plainnat}

\hypersetup{
  colorlinks=true,
  urlcolor=red,
  linkcolor=black,
  citecolor=blue
}

% Code style
\lstdefinestyle{mystyle}{
    basicstyle=\ttfamily\footnotesize,
    breakatwhitespace=false,
    breaklines=true,
    captionpos=b,
    keepspaces=true,
    % numbers=left,
    % numbersep=5pt,
	language=Python,
    showspaces=false,
    showstringspaces=false,
    showtabs=false,
    tabsize=2
}

\lstset{style=mystyle}

% To add numbers in algorithm lines
\LinesNumbered

\renewcommand{\labelenumi}{\alph{enumi}.} % Make numbering in the enumerate environment by letter rather than number (e.g. section 6)

\begin{document}
\begin{titlepage}

\begin{center}
{{\Large{\textsc{Alma Mater Studiorum $\cdot$ University of Bologna}}}}
\rule[0.1cm]{15.8cm}{0.25mm}
\\\vspace{3mm}
%
%
{\Large{Dipartimento di Informatica - Scienza e Ingegneria\\
Artificial Intelligence}}


\end{center}

\vspace{20mm}

\begin{center}{
%
%
	{\LARGE{\textbf{Flatland Challenge}}}}
\end{center}

\vspace{15mm}

{\begin{center}
	 \large{Project Presentation}
\end{center}}

\vspace{32mm} \par \noindent

\begin{minipage}[t]{0.47\textwidth}
%
%
{\large{ Professor \vspace{2mm}\\{\textbf{Andrea Asperti}
}\\\\\\}}
\end{minipage}
%
\hfill
%
\begin{minipage}[t]{0.47\textwidth}\raggedleft{}{
{\large{ Students
\vspace{2mm}\\
\textbf{Alessandro Lombardi\\Fiorenzo Parascandolo} }}}
\end{minipage}

\vspace{31mm}

\begin{center}
Academic Year {2019-2020}
\end{center}

\end{titlepage}

{\tableofcontents}
\thispagestyle{empty}

\newpage
\setcounter{page}{1}

\chapter*{Introduction}

\chapter{Flatland environment}\label{ch:flatland-environment}
\addcontentsline{toc}{chapter}{Flatland}

In order to conceive a solution for this challenge we need to fully understand the environment, how does it work and how do some subtle details influence behind the scenes what we can only perceive through observations and rewards.
The following results have been evaluated on the Python package flatland-rl version 2.1.10.

\section*{The classes RailAgentStatus and EnvAgent}\label{sec:the-classes-railagentstatus-and-envagent}

\textbf{RailAgentStatus} extends Python IntEnum and assumes the following values:
\begin{itemize}
	\item READY\_TO\_DEPART (0) the agent is not in the grid yet (position is None), the prediction is to stay at the starting position.
If a MOVE\_* action is performed during this state it becomes ACTIVE\@.
	\item ACTIVE (1) the agent is in the grid (position is not None) and hasn't reached the target yet, the prediction is the remaining path.
	\item DONE (2) the agent is still in the grid (position is not None) but has already reached the target, the prediction is to stay at the target forever.
	\item DONE\_REMOVED (3) the agent has reached the target and it's removed from the grid.
\end{itemize}

\textbf{Grid4TransitionsEnum} extends Python standard IntEnum and assumes the following values: NORTH (0), EAST (1), SOUTH (2), WEST (3).
\textbf{Grid4TransitionsEnum} is used to indicate absolute directions, related to the environment, like a compass.
Possible usages are storing where the agent is facing or computing legal actions, for example including as observation a one hot encoding of the directions where the agent can move.
\\
\textbf{EnvAgent} class models the agent and encapsulates in its internal state the following attributes:
\begin{itemize}
	\item initial\_position: Tuple[int, int], initial coordinate.
	\item initial\_direction: Grid4TransitionsEnum, the initial agent facing direction.
	\item direction: Grid4TransitionsEnum, the current facing direction.
	\item target: Tuple[int, int], the final coordinate.
	\item moving: bool, True if the agent is in a moving state.
	\item speed\_data: dictionary, containing information about the agent's speed.
	\item malfunction\_data: dictionary, containing information about malfunctions.
	\item status: RailAgentStatus, the current agent status.
\end{itemize}

The speed of an agent contains the keys 'position\_fraction' used as a counter of the percentage of completion of an movement from a cell to another, 'speed' the value between 0 and 1 used to increment the 'position\_fraction' and 'transition\_action\_on\_cellexit' which contains the action to perform on the next cell, if it completes the one in the current step, otherwise in following steps may change multiple times.

The malfunction of an agent contains the keys 'malfunction' which contains how many steps are necessary to fix the agent, 'malfunction\_rate', the mean rate (average number of events in an interval) of the Poisson distribution, 'next\_malfunction' the number of steps the next malfunction will occur and 'nr\_malfunctions' the number of previous malfunctions.

\section{The class RailEnv}\label{sec:the-class-railenv}

From the documentation
\begin{quotation}
	RailEnv is an environment inspired by a (simplified version of) a rail
    network, in which agents (trains) have to navigate to their target
    locations in the shortest time possible, while at the same time cooperating
    to avoid bottlenecks.
\end{quotation}

In the \textit{step} function the number of steps is updated and if the overall task is still uncompleted, for each agent the associated reward is initially put to zero, a malfunction is tried to be induced and the specific step is performed.
The info of the agent are prepared and finally the malfunctions are ``repaired''.
Agents are handled in the order in which are passed.

\subsection{Environment Actions}\label{subsec:environment-actions}
The available actions are:
\begin{itemize}
	\item DO\_NOTHING (0) Default action if None has been provided or the value is not within this list.
If agent.moving is True then the agent will MOVE\_FORWARD\@.
    \item MOVE\_LEFT (1) If agent.moving is False then becomes True.
If it's possible turn the agent left, changing its direction, otherwise if agent.moving is True tries the action MOVE\_FORWARD\@.
    \item MOVE\_FORWARD (2) If agent.moving is False then becomes True.
It updates the direction of the agent and if the new cell is a dead-end the new direction is the opposite of the current.
    \item MOVE\_RIGHT (3) If agent.moving is False then becomes True.
If it's possible turn the agent right, changing its direction, otherwise if agent.moving is True tries the action MOVE\_FORWARD\@.
    \item STOP\_MOVING (4) If agent.moving is True then becomes False.
A penalty will be added.
Stop the agent in the current occupied cell.
\end{itemize}

\begin{algorithm}[H]
	\uIf{agent is in DONE or in DONE\_REMOVED ($1^{\text{th}}$ case)}{
		no reward is computed\\
		\Return{}
	}
	\uIf{agent is in READY\_TO\_DEPART ($2^{\text{th}}$ case)}{
		if the provided action is a MOVE\_* type and the initial cell is free the agent become ACTIVE and is initialized\\
		reward is computed\\
		\Return{}
	}
	\uIf{agent is in malfunction (($3^{\text{th}}$ case))}{
		reward is computed\\
		\Return{}
	}
	\uIf{agent is at the beginning of a cell}{
		update agent.moving considering the observations above depending on the different action types.\\
		\uIf{agent.moving}{
			the wanted action validity is first checked and if it is valid (considering also the possibility to backup from an invalid MOVE\_RIGHT or MOVE\_LEFT to a valid MOVE\_FORWARD) action is stored otherwise agent.moving becomes False and penalties are added, in this process agent.moving and agent.speed\_data['transition\_action\_on\_cellexit'] are updated
		}
	}

	\eIf{agent.moving ($4^{\text{th}}$ case)}{
		Updates the percentage of completion then if it is completely arrived on the next cell, before updating the position, the direction and clears the completion percentage it checks whether the new cell is free. Until the cell remains occupied in the future executions the agent will repeat this process.\\
		reward is computed
	}{
		reward is computed
	}
 	\caption{The \textit{\_step\_agent} algorithm}\label{alg:flatland-agent-loop}
\end{algorithm}
\noindent
\\
\\
Some useful questions:
\begin{itemize}
	\item Can agents stop during the performance of an action between two cells?
Absolutely no, it's like any other action.
	\item Are requested actions during a malfunction ignored?
Yes.
	\item Are requested actions during a not completed movement saved for after execution?
No, because the condition at line 11 is not executed and the conditions in line 16 check whether the cell is free and possibly complete agent data.
Actions are not allowed to change within the cell, each agent can only chose an action to be taken when entering a cell.
This action is then executed when a step to the next cell is valid.
	\item How is possible to understand if an agent is ready to perform an action?
The entry info\_dict["action\_required"] returned by the function \textit{step} of \textbf{RailEnv} contains True for the given agent.
This doesn't mean that the action will be successfully executed due to the presence of malfunctions or blocking agents, in this case info\_dict["action\_required"] will remain True.
info\_dict["action\_required"] is True when an agent is in "READY\_TO\_DEPART" or is in "ACTIVE" and is close to complete a movement (position fraction is near zero).
This means that agents that are arrived to their destination and many agents that are in deadlock (by trying to move they increase their position fraction) can be ignored using this single flag.
	\item Does an agent which has been reached DONE be removed automatically the following step?
Although in rendering its representation may remain, the agent is removed.
It is possible to change this setting modifying the attribute remove\_agents\_at\_target of the class \textbf{RailEnv}.
	\item Does an agent automatically pass from READY\_TO\_DEPART to ACTIVE at the beginning?
No. A MOVE\_* is necessary.
	\item Do collisions occur?
No, agents check if the cell is free before moving.
Deadlocks are possible when two agents are one in front the other without any chance to change path.
\end{itemize}

\subsection{Malfunctions}\label{subsec:malfunctions}
The strategy depends on the passed \textit{malfunction\_generator\_and\_process\_data}.
A strategy can be defined using the class \textbf{MalfunctionParameters} that can be initialized with the parameters to shape the stochasticity of the environment as the malfunction rate, expressed as a probability (Poisson distribution), and minimum and maximum malfunction duration.

\begin{itemize}
	\item A malfunction can occur during the resolution of another?
No.
	\item An agent could have a malfunction during the completion of an action between two cells?
Yes, malfunctions are induced before the step and so before also action completion.
\end{itemize}

\subsection{Speed}\label{subsec:speed}
The different speed profiles (speed is between 0 and 1) can be generated setting the parameter schedule\_generator.
Speed configurations can be build using \textbf{ScheduleGenerator}s.

\subsection{Rewards}\label{subsec:rewards}
The rewards are based on the following values:
\begin{itemize}
	\item invalid\_action\_penalty which is currently set to 0, penalty for requesting an invalid action
	\item \textbf{step\_penalty} which is -1 * alpha, penalty for a time step.
	\item \textbf{global\_reward} which is 1 * beta, a sort of default penalty.
	\item stop\_penalty which is currently set to 0, penalty for stopping a moving agent
	\item start\_penalty which is currently set to 0, penalty for starting a stopped agent
\end{itemize}

The full step penalty is computed as the product between step\_penalty and \\ agent.speed\_data['speed'].
There are different rewards for different situations:

\begin{itemize}
	\item single agents that are in DONE or in DONE\_REMOVED have zero reward ($1^{\text{th}}$ \textit{\_step\_agent} case).
	\item all agents that have finished in this episode (checked at the end of the \textit{step}) or previously (checked at the beginning of the \textit{step}), have reward equal to the global\_reward (when in \textit{step} all agents have reached their target)
	\item full step penalty is assigned when an agent is READY\_TO\_DEPART and in the current turn moves or stay there ($2^{\text{th}}$ \textit{\_step\_agent} case), or when is in malfunction ($3^{\text{th}}$ \textit{\_step\_agent} case).
	\item full step penalty plus the other penalties (invalid\_action\_penalty, stop\_penalty and start\_penalty) when the agent is finishing actions or start new ones ($4^{\text{th}}$ \textit{\_step\_agent} case).
Currently the other penalties are all set to zero.
\end{itemize}

So each train starts counting rewards since the beginning, not since it becomes ACTIVE\@.
Currently it is possible to say that rewards are always full step excluding the end of the episode and the single agents that have finished which have reward equal to 0.

\chapter{Multi-Agent Reinforcement Learning (MARL)}\label{ch:multi-agent-reinforcement-learning}
\addcontentsline{toc}{chapter}{Multi-Agent Reinforcement Learning (MARL)}

This section provides an overview of some useful works and theory behind MARL that we consider useful to suggest approaches or solutions to the Flatland Challenge.

\section{Theoretical background}\label{sec:marl-theoretical-background}

\subsection{Introduction}\label{subsec:introduction}

\begin{quoting}[font=itshape, begintext={"}, endtext={"\citep{MARL_definition}}]
Specifically, MARL addresses the sequential decision-making problem of multiple autonomous agents that operate in a common environment, each of which aims to optimize its own long-term return by interacting with the environment and other agents.
\end{quoting}

MARL algorithms can be divided into three groups~\citep{zhang2019multiagent}:

\begin{itemize}
	\item \textbf{Fully cooperative}, where agents collaborate to optimize a common long-term return.
	\item \textbf{Fully competitive}, where the return of agents usually sum up to zero.
	\item \textbf{Mix of the two}, where both cooperative and competitive agents are involved.
\end{itemize}

We consider the Flatland Challenge a fully cooperative environment since each agent apparently compete to reach faster its destination and gain rails portions but the problem must consider the common time minimization goal.
There exist two closely related theoretical frameworks for MARL~\citep{zhang2019multiagent}:

\begin{itemize}
	\item \textbf{Markov/Stochastic Game}.
All agents share the same state and differently from classical single agent's \textbf{Markov Decision Process (MDP)}, the optimal performance of each agent is controlled not only by its own policy, but also the choices of all other players of the game.
Usually agents share a common reward function but it is possible to have different functions like in the team-average reward setting.
	\item \textbf{Decentralized POMDP (Dec-POMDP)}.
As a MDP can be extended to a \textbf{Partially Observable Markov Decision Process (POMDP)} when information that can be accessed at a given state is incomplete, similarly a Markov Game can be extended to a Dec-POMDP\@.
In multi-agent scenario, an agent may not only depend on the information it has autonomously gathered, it will also be influenced by the choices of other agents, which are partially observable.
In Dec-POMDP each agent has its own local observation of the system state, that without other agents' observations, leads to the impossibility to maintain a global belief state.
To overcome this problem, agents can exploit levels of coordination among them to obtain the full observability of a state by combining the individual observations from each member~\citep{castaneda}.
Dec-POMDP approaches are usually considered more difficult to solve than others, especially when the number of agents is greater than two.
	\item \textbf{Extensive-Form Game} inspired from computational game theory, it handles imperfect information.
It is usually used in mixed or competitive environments.
\end{itemize}

In the Flatland environment a state is surely represented by the positions and orientations of each agent in the map, since the railroad is static and agents influence it only by moving and interrupting paths.
Flatland environment allows to personalize how agents perceive the world implementing custom observations.

The involvement of Deep Learning to tackle the problem of MARL defines a new specific subject called \textbf{Multi-agent Deep Reinforcement Learning (MADRL)}.
~\citep{Hernandez_Leal_2019} presents four categories of recent MADRL works:
\begin{itemize}
	\item \textbf{Analysis of emergent behaviors}, in general, they do not propose learning algorithms, their main focus is to analyze and evaluate DRL algorithms in a multi-agent environment.
	\item \textbf{Learning communication}, they study communications techniques to share information.
	\item \textbf{Learning cooperation}, they directly explore approaches based on actions and observations to build multi-agent systems.
	\item \textbf{Agents modeling agents}, they study how agents reason about others to fulfill a task.
\end{itemize}

\subsection{Challenges}\label{subsec:challenges}

MARL frameworks inevitably adds many challenging problems on the single-agent scenario.
\begin{quoting}[font=itshape, begintext={"}, endtext={"\citep{Hernandez_Leal_2019}}]
Learning in multiagent settings is fundamentally more difficult than the single-agent case due to the presence of multiagent pathologies, e.g., the moving target problem (non-stationarity), curse of dimensionality, multiagent credit assignment, global exploration, and relative overgeneralization.
\end{quoting}

Below follow some problems that may affect the Flatland Challenge.

\subsubsection{Non-Stationarity}

\begin{quoting}[font=itshape, begintext={"}, endtext={"\citep{papoudakis2019dealing}}]
In Markov games, the state transition function $T$ and the reward function of each agent $r_i$ depend on the actions of all agents.
During the training of multiple agents, the policy of each agent changes through time.
As a result, each agents’ perceived transition and reward functions change as well.
Single-agent RL procedures which commonly assume stationarity of these functions might not quickly adapt to such changes.
\end{quoting}

In a single-agent environment, an agent is concerning only the outcome of its own actions.
In a multi-agent scenario, an agent observes not only the outcomes of its own action but also the behavior of other agents.
Agents may interact with each other and learn concurrently leading to a continuous reshape of the environment and to non-stationarity~\citep{zhang2019multiagent}.
For example the classical DQN does not provide working solutions, some derivations have been proposed to deal with this problem such as \textbf{Deep Repeated Update Q-network (DRUQN)}~\citep{castaneda}, \textbf{Deep Loosely Coupled Q-network (DLCQN)}~\citep{castaneda} and \textbf{ multi-agent concurrent DQN}.
Other techniques to adapt classical experience replay to multi-agent environment have been proposed such as \textbf{Hysteretic-DQN (HDQN)} and \textbf{Lenient-DQN (LDQN)}~\citep{Nguyen_2020}.

There are different ways to tackle the non-stationary problem, as illustrated by~\citep{papoudakis2019dealing}.
\begin{itemize}
	\item \textbf{Centralized Critic Architecture} based on an actor-critic algorithm.
The critics' training is centralized and has access to the observations and actions of all agents, while the actors' training is decentralized.
An example is the \textbf{Multi-Agent Deep Deterministic Policy Gradient (MADDPG)} algorithm.
In MADDPG each agent uses a centralized critic and a decentralized actor.
Since the training of each agent depends on the observations and actions of all the other agents, each agent perceives the environment as stationary.
	\item \textbf{Decentralized Learning} Techniques using self-play.
	\item Opponent Modelling.
	\item \textbf{Meta-Learning}.
	\item Communication.
Either accessing hidden layers as \textbf{CommNet} or feeding other agents' neural networks as \textbf{Reinforced Inter-Agent Learning (RIAL)}.
\end{itemize}

A popular alternative approach to MARL is \textbf{Independent Learning}, in which each agent independently learns its own policy, treating other agents as part of the environment.
While this method avoid some scalability problems and has been successfully used in practice it introduces a non-stationary environment from the point of view of each agent.

\subsubsection{Partial observability}
As mentioned, in the single-agent scenario this type of problem is usually modelled with a POMDP\@.
\textbf{Deep Recurrent Q-Networks (DRQN)} proposed using recurrent neural networks, in particular, Long Short-Term Memory (LSTMs) cells in DQN, to introduce a memory capability.
An extension of DRQN for multi-agent environments is \textbf{Deep Distributed Recurrent Q-network (DDRQN)}~\citep{Nguyen_2020}.
Another technique to deal with partial observability is \textbf{Deep Recurrent Policy Inference Q-network (DRPIQN)} learned by adapting network’s attention to policy features and their own Q-values at various stages of the training process.
Methods to address partial observability in Markov Games usually involve communication (RIAL and \textbf{Differentiable Inter-Agent Learning (DIAL)}) or parameter sharing (\textbf{PS-DQN}, \textbf{PS-DDPG}, \textbf{PS-A3C} and \textbf{PS-TRPO}).
Alternatively to the Markov Game, the Dec-POMDP can be used to model this type of scenario in a more classical \textbf{centralized learning for decentralized execution} fashion or in a more modern \textbf{decentralized learning for decentralized execution} way.

\subsubsection{Scalability}
To handle non-stationarity, each individual agent may need to account for the joint action space, whose dimension increases exponentially with the number of agents, this is  also referred to as the combinatorial nature of MARL\@~\citep{zhang2019multiagent}.
Many methods have been proposed to tackle this problem, one of them is the extension of \textbf{Curriculum Learning} for a multi-agent scenario.

\subsubsection{Information Structures and Training Schemes}~\newline
In the single-agent case is easier to understand what information is visible to the agent.
In Markov games is sufficient to observe the current state, while on extensive-form games agents may need to recall the history of past decisions.
In addition agents struggle to fully access information like rewards and policies of other agents, increasing the non-stationarity viewed by by individual agents~\citep{Nguyen_2020}~\citep{zhang2019multiagent}.
These considerations led to the development of different training schemes such as centralized learning for decentralized execution, which originated from the works on the Dec-POMDP setting and has been widely adopted in recent MADRL works, and \textbf{fully decentralized}.
The former has become a standard as simulators are usually involved in MARL training, there are different types of communications between agents and the central controller such as centralized learning, concurrent learning and parameter sharing\citep{Nguyen_2020}.
Usually decentralized settings may allow some sort of communication to address the non-convergence issue typical of the independent learning, this strategy is also referred as \textbf{decentralized setting with networked agents}.

\subsubsection{Multi-agent credit assignment}
% TODO

\chapter{Original work}\label{ch:original-work}
\addcontentsline{toc}{chapter}{Original work}

This section describes the implementation and the testing of some Deep Reinforcement Learning algorithms explored in this work.
We developed the following work using Python and Pytorch.

\section{Deadlocks}

In Flatland the deadlocks are a truly catastrophic event because the agents involved can no longer move and are an obstacle to those free to navigate for the rest of the episode. 
Deadlocks detection is an additional challenge whereas there is still no algorithm for this particular task that can be performormed in polynomial time, however an estimate of the occurence of these events is valuable information both for performance analysis described in~\ref{subsec:metrics} and for Rewards Shaping described in ~\ref{subsec:rewards2}.
The algorithm proposed is not complete: if a situation is detected as a deadlock, it is really a deadlock, but it is also possible that a deadlock is not recognized as a deadlock at all.

\begin{algorithm}[H]
\SetKwFunction{FCheckDeadlocks}{CheckDeadlocks}

\For{$agent$ in range($n\_agents$)}{
	\uIf{agent is ACTIVE and not(deadlocks[agent])}{
		$deadlocks[agent]$ = CheckDeadlocks(agent)
	}
}

\SetKwProg{Fn}{Function}{:}{\KwRet}
\Fn{\FCheckDeadlocks{$agent$}}{
	    $agent\_2 = check\_next\_pos(agent)$ \\ 
	    if agent is active it is checked whether the cell he will occupy, taking into account his direction, there is another agent. \\
	    If its direction is not valid, i.e. if the cell it is facing is not part of the railway, $check\_next\_pos$ searches in space for possible transitions. \\
            \uIf{$agent\_2$ is $None$}{
 		\KwRet\ False; \\
		if there is no agent in the cell he will take care then $agent$ is not in deadlock.
	    }
	    \uIf{$deadlocks[agent\_2]$}{
		\KwRet\ True; \\
		if agent\_2 was already deadlocked agent is deadlocked too. 
	    }
	    \uElse{
		 $deadlocks[agent\_2]$ = CheckDeadlocks(agent\_2) \\
		 If agent\_2 is not deadlocked until the current step, it is checked if it is deadlocked.
	    }
	    \If{$deadlocks[agent\_2]$}{
		\KwRet\ True \\
		if in the current step agent\_2's deadlock is detected then $agent$ is also in deadlock
            }

	    \KwRet\ False
}
\caption{The \textit{deadlocks\_detection} algorithm}
\end{algorithm}

\section{Metrics and evaluation}\label{sec:metrics-and-evaluation}

\subsection{Metrics}\label{subsec:metrics}
We implemented a \textbf{StatsWrapper} to compute and print the metrics to evaluate the algorithm's performance:

\begin{itemize}
\item \textbf{normalized\_score} is the sum of the rewards accumulated by all agents during the episode divided by the worst score obtainible, computed as the product between the number of agents and the maximum number of steps in the episode. In the worst case, infact, all agents do not reach their destination, therefore for each step they get a negative reward. 
\begin{equation}{score \over {max\_steps \cdot n\_agents}}\end{equation}
\item \textbf{accumulated\_normalized\_score} is the mean of \textbf{normalized\_score} obtained up to that point. 
\begin{equation}{\sum {normalized\_score} \over N}\end{equation}
\item \textbf{completion\_percentage} is the percentage of agents who reached their destination in the episode.
\begin{equation}{100 \cdot {tasks\_finished \over n\_agents}}\end{equation}
\item \textbf{accumulated\_completion} is the mean of \textbf{completion\_percentage} obtained up to that point.
\begin{equation}{\sum {completion\_percentage} \over N}\end{equation}
\item \textbf{deadlocks\_percentage} is the percentage of deadlocks that occured in the episode.
\begin{equation}{100 \cdot {n\_deadlocks \over n\_agents}}\end{equation}
\item \textbf{accumulated\_deadlocks} is the mean of \textbf{deadlocks\_percentage} obtained up to that point.
\begin{equation}{\sum {deadlocks\_percentage} \over N}\end{equation}
\end{itemize}

The \textbf{StatsWrapper} also provides the probability distribution of the actions taken each episode.

\subsection{Evaluation}
%TODO: wandb

\section{Observations, actions and rewards}\label{sec:observations-and-actions}

This section provides some details on how observations and actions have been modeled in the various implementations.

\subsection{Observations}\label{subsec:observations}

Flatland environment provides three basic observations to get started: \textbf{Global}, \textbf{Local Grid}, \textbf{Local Tree}.
Due to the greater scalability we have done the experiments mostly using \textbf{Local Tree} as it depends on some parameters which are independent of the size of the grid and the number of agents.
For the latter the observation vector is composed of 4 sequential parts, corresponding to data from the up to 4 possible movements in a \textbf{RailEnv}.
The possible movements are sorted relative to the current orientation of the agent, rather than NESW as for the transitions.
The order is \textbf{left}, \textbf{forward}, \textbf{right}, \textbf{back}.
Each branch data is organized as:

\begin{lstlisting}[label={lst:tree-obs}]
[root node information] +
[recursive branch data from 'left'] +
[recursive from 'forward'] +
[recursive from 'right] +
[recursive from 'back']
\end{lstlisting}

Each node information is composed of 9 features:

\begin{enumerate}
\item [1:]If own target lies on the explored branch the current distance from the agent in number of cells is stored.
\item [2:]If another agent’s target is detected, the distance in number of cells from the current agent position is stored.
\item [3:]If another agent is detected, the distance in number of cells from the current agent position is stored.
\item [4:]Possible conflict detected (This only works when we use a predictor and will not be important in this tutorial)
\item [5:]If an unusable switch (for the agent) is detected we store the distance.
An unusable switch is a switch where the agent does not have any choice of path, but other agents coming from
\item [6:]This feature stores the distance (in number of cells) to the next node (e.g.\ switch or target or dead-end)
\item [7:]Minimum remaining travel distance from this node to the agent’s target given the direction of the agent if this path is chosen
\item [8:]Agent in the same direction found on path to node
\item [9:]Agent in the opposite direction on path to node
\end{enumerate}

\textbf{TreeObsForRailEnv} depends on three hyperparameters:

\begin{enumerate}
\item [1.] \textbf{observation\_tree\_depth} represents the depth of the observation tree.
\item [2.] \textbf{observation\_radius} is used in the normalization phase.
\item [3.] \textbf{observation\_max\_path\_depth} is the shortest-path predictions for agents in the environment.
\end{enumerate}

\subsection{Actions}\label{subsec:actions}
As mentioned before~\ref{subsec:environment-actions} the Flatland environment provides for each agent five different actions.
The DO\_NOTHING is not necessary to reach a solution, because the agent can continue moving forward deciding each time the action MOVE\_FORWARD or stop using STOP\_MOVING\@.
As we think that further from being useless it may also damage the overall performance we decided to consider its removal.
It has already mentioned the ambiguity of the actions MOVE\_LEFT and MOVE\_RIGHT where they are forbidden, it is natural to conclude that the agents may learn bad policies that maps these actions to the same effect of the MOVE\_FORWARD action.
We observed that this phenomenon is very common due to the presence of long straight paths where the agent is allowed only to stop or move forward and concluded that even stopping in the middle of the rail does not have much sense because agents still stop when other agents block their way due to deadlocks, different speeds or malfunctions.
For this reason we considered the possibility to force agents to only decide and learn in switches, where multiple actions are allowed and agents may learn to give way to other agents, avoid deadlocks, reach the target and more.
This considerations lead to skipping a lot of choices during learning and deploy \textbf{Action Masking} to avoid illegal actions.
Some studies~\citep{ppo_action_masking} have proposed to deploy action masking to avoid the selection of multiple actions when they are not necessary, for example in the game Dota 2 the full action space is of 1,837,080 dimensions \href{https://cdn.openai.com/dota-2.pdf}{OpenAI Dota2}.
As mentioned in~\ref{subsec:rewards2} a very common strategy to address invalid actions is applying negative rewards, but this also requires the agent to explore the actions and understand how to map actions to the possibility of applying them.
During this period it is possible that the agent converges to a wrong policy.
Invalid action masking helps to avoid sampling invalid actions by ``masking out'' the network outcomes corresponding to the invalid actions.
This is usually accomplished by replacing the values of the actions to be masked by a large negative number like $-1 \cdot 10^8$.
This operation actually change the gradient calculation for the actor network's parameters, in particular gradients of the masked logits become zero, but still maintains differentiability as the masking leaves unchanged values (identity function) or introduces constants.
% TODO: action skipping, pytorch impl and tests

~\ref{subsec:rewards2} analyze an alternative approach to action masking based on rewards.

\subsection{Rewards}\label{subsec:rewards2}

As mentioned before~\ref{subsec:rewards} the Flatland environment provides a basic rewards system which in the current implementation we can briefly describe in this way:
\begin{itemize}
	\item Every step agent receives a negative reward proportionate to his speed if he has not reached his destination.
	\item Each agent receives a reward equal to 0 if he has reached his destination.
	\item If all agents have reached their destination they receive a reward equal to 1.
\end{itemize}
We think that this reward system does not represent all the complexity of the problem because there is not much distinction between the states that agents may be in while navigating in the environment.
Agents must basically learn two behaviors:
\begin{itemize}
	\item Reach their destination in the shortest time possible.
	\item Avoid collisions with other agents.
\end{itemize}
This is not neccessarily the order.
Let's consider an environment with 3 agents, in this case the main behavior is the first because the probability of a collision si not very relevant, but if we consider the same environment with 10 agents the skill to avoid deadlocks is decisive for performance.
According to the Flatland's rewards system there is not difference between being deadlocked and navigating the map without reaching destination from an agent's point of view in terms of rewards.
In order to stimulate the learning of the desired behaviors we have tried to modify the Flatland's rewards system, a method that in literature is called \textbf{Rewards Shaping}.
Crafting rewards is not easy beacuse as a consequence we could get a cobra effect:

\begin{quoting}[font=itshape, begintext={"}, endtext={ \footnote{https://medium.com/@BonsaiAI/deep-reinforcement-learning-models-tips-tricks-for-writing-reward-functions-a84fe525e8e0}}]
Historically, the government tried to incentivize people to assist them in ridding the area of cobras.
If citizens brought in a venomous snake they had killed, the government would give you some money.
Naturally, people started breeding venomous snakes.
\end{quoting}
Therefore, sometimes, this method may cause an undesirable effect: stimulating the learning of one behavior can cause the learning of another wrong.
Anyway  we have implemented a \textbf{RewardsWrapper} to be able to choose flexibly during the train phase whether to use the rewards shaped or the standard Flatland's reward system.
In the first case it is possible to choose how to shape the rewards by choosing the following parameters:
% TODO: Uniform reward for agent with different speed

\begin{itemize}
	\item \textbf{$invalid\_action\_penalty$}: It is used to penalize the agents who have chosen an invalid action, which is calculated considering the position and direction of the agent within the environment, subtracting the assigned value to the reward associated with the agent according to the standard Flatland's rewards system.
	\item \textbf{$stop\_penalty$}: It is used to penalize more agents who have chosen the STOP\_MOVING action by subtracting the value assigned to the reward computed in the previous step.
	\item \textbf{$deadlock\_penalty$}: It is used to penalize agents in deadlocks by subtracting the value assigned to the reward associated with the agent calculated in the previous step.
	\item \textbf{$shortest\_path\_penalty\_coefficient$}: It is used to penalize agents who are moving away from their target by multiplying the value assigned to the reward associated with the agent calculated in the previous step.
	\item \textbf{$done\_bonus$}: It is used to reward agents who have arrived at their destination by assigning them a positive reward.
\end{itemize}
Let's consider for example this set of parameters:

\begin{lstlisting}[label={lst:rewards-params}]
"reward_shaping": True,
"stop_penalty": -0.0,
"invalid_action_penalty": -1.0,
"deadlock_penalty": -3,
"shortest_path_penalty_coefficient": 1.5,
"done_bonus": 1.0,
\end{lstlisting}

suppose an agent moves with unit speed and in a certain instant he chooses an invalid move due to his position which distances him from his target.
According to the description given above, the reward will be calculated as follows:
\begin{displaymath}{reward\_shaped = standard\_reward + invalid\_action\_penalty = -1 -1 = -2}\end{displaymath}
\begin{displaymath}{reward\_shaped = shortest\_path\_penalty\_coefficient * rewards\_shaped = 1.5 * (-2)= -3}\end{displaymath}

\section{PS-PPO}\label{sec:ps-ppo}

The algorithm is inspired by the work~\citep{ps-ppo_paper} which extends three classes of single-agent Deep Reinforcement Learning (DQN, DDPG and TRPO) to cooperative multi-agent systems.
In their work they explore the task of learning cooperative tasks in partially observable environments using an implicit communication protocol based on parameters sharing.
% TODO: their communication (id agent)
They illustrate that centralized approaches, based on mapping the joint observation of all the agents to a joint action, suffer from the exponential growth in the observation and actions spaces with the number of agents, while concurrent and independent techniques suffer from the non-stationarity of the multi-agents environments and the lack of communication.
Parameter sharing represents an ideal trade off to tackle the major challenges of multi-agent systems and provide a scalable framework.

\subsection{Algorithm}\label{subsec:algorithm}

The authors suggest as a policy gradient algorithm the \textbf{Trust Region Policy Optimization (TRPO)} while in this work we present an alternative version based on \textbf{Proximal Policy Optimization (PPO)}.
We will not cover here all the details and the state of the art of the mentioned algorithms in single-agent setting, but they can be found in the references~\citep{trpo} and~\citep{ppo}.
The basic principle of Policy Gradient methods relies on the concept of gradient ascent to follow policies with the steepest increase in rewards.
However, this simple setting may lead the algorithm to get overconfidence and make bad moves that ruin almost irreversibly the progress of the training.
TRPO has been proposed to solve this issue by introducing the concept of guaranteed monotonic improvement.
Theoretically, TRPO can guarantee a policy improvement as long as it optimizes the local approximation within a trusted region, creating at each update a better policy.
PPO was conceived to improve the policy-based techniques by introducing an algorithm that attains the data efficiency and reliable performance of TRPO, while using only first-order optimization, as one of the most remarkable drawback of TRPO is its complexity.
PPO relies on clipped probability ratios, which forms a pessimistic estimate of the performance of the policy, and data sampled from applying for several time steps the policy to obtain a so called trajectory.
Trajectories are finally used to optimize the policies in multiple epochs of mini-batch updates.
% TODO: our communication (uniform id agent)

\subsection{Implementation}\label{subsec:implementation}

This part does not describe some implementation details to exclusively show our work but also to underline some important aspects from the practical point of view which have been pointed out by the literature~\citep{ppo_implementation_1}~\citep{ppo_implementation_2} to be crucial for the overall algorithm performance.
In particular they found significantly beneficial to initialize the policy network so that the action distribution has zero mean, a low standard deviation and is independent from the observation.
Our studies follow the path of these helpful outcomes taking into account that our action and observation spaces are discrete and not continuous and that our environment is multi-agent.
The majority of the following details are not even mentioned in the original PPO paper but have been discovered directly implemented in the OpenAI Baselines~\citep{ppo_baselines} as mentioned in the article~\citep{ppo-32-implementations-details}.

\subsubsection{Network architecture}

PPO belongs to the family of \textbf{actor-critic algorithms}, it requires the design of two different neural networks: the critic, which estimates the value function used as a baseline, and the actor, which updates the policy distribution in the direction suggested by the critic.
Some actor-critic implementations make parts of the actor and critic networks shared to decrease computational costs while others prefer to avoid this because it makes harder training and tuning of hyperparameters.
This is because the norm of gradients flowing back from the actor gradients and the critic gradients may be at completely different scales causing difficulties during calibration.
First and last layers sizes are bound to the observation and action dimensions, while hidden layers may have a different one, cited studies have discovered that in many environments a wider value MLP network is better and Tanh activation functions are more preferred than ReLU\@.
Neural networks' weights are initialized using orthogonal initialization while biases are initially set to zero.
% TODO: network generation pytorch

\begin{quoting}[font=itshape, begintext={"}, endtext={"\citep{ppo_implementation_2}}]
Interestingly, the initial policy appears to have a surprisingly high impact on the training performance.
The key recipe appears is to initialize the policy at the beginning of training so that the action distribution is centered around 0 regardless of the observation and has a rather small standard deviation.
This can be achieved by initializing the policy MLP with smaller weights in the last layer so that the initial action distribution is almost independent of the observation and by introducing an offset in the standard deviation of actions.
\end{quoting}

From the quote is possible to observe the importance of network weights initialization especially in the last layer of the policy network.
It is important to specify that in their case the action distributions (Gaussians) are continuous and are centered on 0 because action spaces are normalized between -1 and 1.
On the other hand, Flatland's action space is discrete and we modelled it with a Categorical Distribution, a generalization of a Bernoulli distribution for a categorical random variable.
By rescaling the last policy layer's weights of the policy network with $100\times$ smaller values the action distribution is more balanced between all possible actions for the first iterations, leading to better initial stability and improving exploration.

\begin{lstlisting}[label={lst:psppo-net-init}]
def weights_init(m):
if isinstance(m, nn.Linear):
	torch.nn.init.orthogonal_(m.weight, np.sqrt(2))
	torch.nn.init.zeros_(m.bias)

with torch.no_grad():
	self.critic_network.apply(weights_init)
	self.actor_network.apply(weights_init)

	# Last layer's weights rescaling
	list(self.critic_network.children())[-1].weight.mul_(train_params.last_critic_layer_scaling)
	list(self.actor_network.children())[-1].weight.mul_(train_params.last_actor_layer_scaling)
\end{lstlisting}

Neural networks are trained using Adam optimizer.
References explore more alternatives and approaches involving the optimizer and its hyperparameters as learning rate decay and momentum but we didn't cover them.
% TODO: target network

\subsubsection{Training setup}\label{subsubsec:training-setup}

The training setup is similar to the single-agent one but introduces more complexity to deal with multiple agents.
In particular in the first part each agent interact with the environment gathering some experience, called also trajectory.
Each agent's trajectory includes for each time step the related observation and the action performed, originally rewards and information about the termination of the problem of each agent were joined together to form uniques values, but after some tests we realized that this was not correct and we reverted rewards and termination information to a single agent's perspective experiencing a great performance improvement.
During this modifications we accidentally discovered an alternative interpretation of agent's trajectory which does not include the episode ending as a terminal condition but only the reaching of a destination, also across different episodes.
Apparently this seems an incorrect application, as at the beginning of a new episode the agent resets its position and starts a new path which is not related to the previous one.
As a consequence, terminal states influence especially the advantage computation and longer trajectories may lead to consider with less importance the estimations of the value of latter parts of the trajectory due to the presence of the discount factor.%TODO conclude and ref tests
\\
\\
Trajectories' lengths particularly depend on the chosen horizon, deciding how much experience should an agent collect before updating the policy is crucial.
It is important to consider that PPO is an on-policy algorithm, every time experience has been collected and used to train the policy new experience must be gathered and the old is thrown away.
This behaviour is strictly related to a common RL feature called \textbf{sample efficiency}.
Off-policy algorithms such as DQN are more sample efficient because can store huge quantities of information and use them more easily and efficiently.
We hypothesize that horizon must depend on the specific environment, and the implications in Flatland are even deeper as the environment can drastically change its complexity. % TODO: horizon depends on action skipping too (should decrease)
After some experiments we observed that higher values of horizon tends to perform better and we have found similar comments on other works, including arguments on implications of the value to assign to the decaying factor used to calculate the advantage~\ref{subsubsec:advantage-estimation}.\\
\\
One of the greatest advantage of PPO, thanks to PPO's simpler objective, over the other policy-based algorithms is the possibility to perform multiple epochs on the collected data.
The classical approach is to repeat for a certain number of epochs the training on the same mini-batches obtained preserving the temporal order of the individual transitions inside the overall experience.
But this approach usually leads to another typical problem of policy gradient methods: if chunks of experiences are small and are temporally ordered, learning uses similar samples of data causing instability as may occur large updates with gradients pointing to the same direction.
To avoid this problem data can be shuffled in different ways.
Trajectories can be shuffled to change the order in which mini-batches are used in each epoch or transitions can be shuffled precomputing the advantages at the beginning of the iteration (OpenAI Baseline's style) or recomputing them at the beginning of each epoch.
We decided to follow the OpenAI Baseline's implementation, leading to higher diversity of data in each mini-batch but a less updated advantage, but to leave also the option to avoid any type of shuffling, observing better performance in the first case.
For each agent the learning phase is repeated in each epoch using batches of length according to the size of batches set as an hyperparameter.
Increasing batch size should increase the speed of learning while adjusting the horizon has a great impact on the final result.

\subsubsection{Normalization and clipping}

\begin{quoting}[font=itshape, begintext={"}, endtext={"\citep{ppo_implementation_2}}]
Always use observation normalization and check if value function normalization improves performance.
Gradient clipping might slightly help but is of secondary importance.
\end{quoting}

We normalize advantages in each mini-batch by subtracting their mean and dividing by their standard deviation for the policy loss.
Obviously as it represents a core part of the PPO algorithm the surrogate objective function is clipped using the hyperparameter epsilon.
The references mention also the usage of a clipping function for the value function loss (as present in the original PPO implementation) with poor results.
Gradient clipping has been adopted in many neural networks, especially recurrent neural networks to avoid gradient explosion by rescaling it when greater than the hyperparameter "max\_grad\_norm" which is also the norm of the new gradient vector.
\begin{lstlisting}[label={lst:gradient-clipping}]
if self.max_grad_norm is not None:
	nn.utils.clip_grad_norm_(self.policy.parameters(), self.max_grad_norm)
\end{lstlisting}

\subsubsection{Advantage estimation and loss}\label{subsubsec:advantage-estimation}

The references show how the \textbf{Generalized Advantage Estimation (GAE)} outperform the other techniques such as \textbf{N-step} and \textbf{V-trace}.

The $Q$ value can be estimated in the following way:
\begin{equation}
	V_t^{(N)} = \sum_{i = t}^{t + N - 1} \gamma^{i - t}r_i + \gamma^{N}V^{\pi}(s_{t + N})\label{eq:N-step}
\end{equation}

where $V^{\pi}$ is a value function approximator, $N$ controls the bias–variance tradeoff of the estimator and $\gamma$ is the discount factor.
$V^{\pi}$ tends to be biased because it is computed with a function approximator but has lower variance than the actual summed rewards since they are obtained from a single trajectory.
Bigger $N$ values results in an estimator closer to empirical returns and induce less bias and more variance.
Variance of the actual rewards typically increases the more steps away from the current step $t$ are taken, while close to $t$, the benefits of using an unbiased estimate may outweigh the variance introduced.
As $N$ increases, the variance in the estimates will likely start to become problematic, and switching to a lower variance but biased estimate can be better.
The advantage is measured subtracting from $V_t^{(N)}$ the estimated value $V^{\pi}(s_t)$.

\begin{equation}
	A_t^{(N)} = V_t^{(N)} - V^{\pi}(s_t)\label{eq:N-step-advantage}
\end{equation}

GAE was proposed as an improvement over the N-steps result~\citep{gae}.

\begin{quoting}[font=itshape, begintext={"}, endtext={"\citep{graesser2019foundations}}]
It addresses the problem of having to explicitly choose the number of steps of returns, n.
The main idea behind GAE is that instead of picking one value of n, we mix multiple values of n.
That is, calculate the advantage using a weighted average of individual advantages calculated with n = 1, 2, 3, \dots, k.
The purpose of GAE is to significantly reduce the variance of the estimator while keeping the bias introduced as low as possible.
\end{quoting}

GAE is defined in the following way:

\begin{equation}
	V_t^{(GAE)} = (1 - \lambda)\sum_{N > 0} \lambda^{N - 1}A_t^{(N)}\label{eq:GAE}
\end{equation}

where $0 < \lambda < 1$ is a hyperparameter controlling the bias–variance trade-off.
\\
\\
The total loss function is composed by three elements:
\begin{itemize}
	\item The surrogate loss obtained from the minimum between the clipped and the unclipped objectives.
	\item The value loss obtained from the training of the critic network as a state-value function, used to predict values that are needed for the advantages computation in the surrogate loss.
The value loss is computed as a squared error loss. % TODO check
	\item Entropy computed from the action categorical distribution to ensure sufficient exploration.
\end{itemize}

In our implementation these three elements are added together and used to compute the gradients with the Pytorch function \textit{backward()} and then optimize the networks.
The value loss and the entropy in the sum are multiplied by their respective coefficients.
%TODO: comments on coefficients and loss values


\subsection{Testing}\label{subsec:testing}

PS-PPO hyperparameters
\begin{itemize}
	\item Network Architecture
	\begin{itemize}
		\item "shared": False
		\item "critic\_mlp\_depth": 3
		\item "last\_critic\_layer\_scaling": 0.1
		\item "actor\_mlp\_width": 128
		\item "actor\_mlp\_depth": 3
		\item "last\_actor\_layer\_scaling": 0.01
		\item "learning\_rate": 0.001 (Adam learning rate)
		\item "adam\_eps": 1e-5 (Adam $\epsilon$ value)
		\item "activation": "Tanh" ("Tanh or "ReLU")
	\end{itemize}
	\item Training setup
	\begin{itemize}
		\item "n\_episodes": 2500
		\item "horizon": 512 (Common values are 512, 1024, 2048, 4096)
		\item "epochs": 8
		\item "batch\_size": 32 (Common values are 64, 128, 256)
		\item "batch\_mode": "shuffle" ("shuffle" or "normal")
	\end{itemize}
	\item Policy and advantage
	\begin{itemize}
		\item "advantage\_estimator": "gae" ("gae" or "n-steps")
		\item "lmbda": 0.95
		\item "entropy\_coefficient": 0.01
		\item "value\_loss\_coefficient": 0.001
	\end{itemize}
	\item Normalization and clipping
	\begin{itemize}
		\item "discount\_factor": 0.99 (Common values are 0.95, 0.97, 0.99, 0.999)
		\item "max\_grad\_norm": 0.5 (Gradient clipping)
		\item "eps\_clip": 0.2 (Objective clipping)
	\end{itemize}
	\item Action Masking and Skipping
	\begin{itemize}
		\item "action\_masking": False
		\item "allow\_no\_op": True
		\item "action\_skipping": False
	\end{itemize}
\end{itemize}

%TODO: horizon and batch, reward shaping, masking, skipping, allow no op

\section{D3QN}\label{sec:d3qn}

\textbf{Deep Q Networks (DQN)} have been proposed as the union of Deep Learning with Q-Learning to deal with complex and high dimensional environments such as video games and robotics.
In this setting the neural network plays as a predictor from states to Q values, substituting a less scalable table of values.
From the initial work~\citep{dqn} many supplements have been proposed to improve different parts of the original algorithm.
\textbf{Dueling Double Deep Q Networks (D3QN)} represents a combination of many of them.
A very wide overview of the main techniques applied to DQN can be found in~\citep{human-level}.

\subsection{Algorithm}\label{subsec:algorithm2}

The first D of D3QN stands for Dueling, an idea proposed in~\citep{dueling} which splits the neural network in two parts, one which computes an estimate of the values as usual and another the advantage of each action to understand which states are valuable without exploring the outcomes of each possible action.
For example in the Flatland environment agents does not perform actions which substantially influence the environment, as many of them are moving forward.
In this case, the idea is to increase the training phase and allow the controller to detect and learn how to act in critical positions such as switches.
At the end, advantages and values are combined to obtain Q values.
The second D stands for Double, which refers to the idea of using two separate neural networks during training, one, called the primary network, is used to compute Q values used to choose the optimal action, while the other, called the target network, is used to estimate the Q value of the chosen action in the given state.
% TODO: finish
\\
\\
The mentioned improvements have been proposed in a single-agent scenario, but as already discussed in chapter~\ref{ch:multi-agent-reinforcement-learning} many adaptations to classical RL algorithms have been proposed for multi-agent setting.
The following part will continue describing a soft adaptation to multi-agent scenario which relies mostly on the classical implementation while~\ref{subsubsec:fingerprints} describes more specific tricks.

\subsection{Implementation}\label{subsec:implementation2}

This section describes the most remarkable details of the algorithm implementation.

\subsubsection{Network architecture}

%TODO value-adv aggregator, eps decay

\subsubsection{Training setup}

In each episode the agents decides an action whenever they are ready to perform an action, namely they are not already moving from a cell to another, this may also include agents in deadlock which are involuntarily accumulating movements they will never perform, and are not arrived to destination.
Agents that not decide an action perform a DO\_NOTHING action.
Actions are stored and used to step the environment, trajectories are saved in the memory.
When an agent decide the action has also th possibility to start the policy training which occurs whenever the memory has reached a certain size and every interval of episodes.
Neural networks are updated with sampled batches of transitions stored in the memory, the different mechanisms of sampling and types of memories are described in~\ref{subsubsec:memory}.
%TODO: check

\subsubsection{Memory}\label{subsubsec:memory}

As mentioned in~\ref{subsubsec:training-setup} one of the most important aspect to consider in a DRL algorithm is the nature of the input given to learn policies or value functions.
Value based methods, such as D3QN, use experience more easily, allowing agents to collect it and reuse it in future learning iterations, this technique is called \textbf{Experience Replay (ER)}.
Samples of batches are drawn from the pool uniformly.
This approach introduce two main benefits: data is reused (sample efficiency) and correlations between randomly sampled steps is lower than using online data.

\begin{quoting}[font=itshape, begintext={"}, endtext={"\citep{human-level}}]
we used a biologically inspired mechanism termed experience replay that randomizes over the data, thereby removing correlations in the observation sequence and smoothing over changes in the data distribution.\\
\[\dots\]\\
In the future, it will be important to explore the potential use of biasing the content of experience replay towards salient events, a phenomenon that characterizes empirically observed hippocampal replay, and relates to the notion of "prioritized sweeping" in reinforcement learning.
\end{quoting}

\textbf{Prioritized Experience Replay (PER)} introduced in~\citep{prioritized}, extends the uniform ER by learning to replay memories where the real reward significantly diverges from the expected reward, letting the agent adjust itself in response to developing incorrect assumptions.
PER implementation can be described in three parts:
\begin{itemize}
	\item The \textbf{SumTree} data structure.
SumTrees are binary trees where each parent node contains the sum of its children.
In this context the tree is used to store data and related priorities inside the leaves while the other nodes are involved in updating more efficiently priorities to speed up samplings and insertions.
	\item The part associated to insertion of new data.
Before inserting new data, an initial priority must be computed and for each new transition a prediction of the Q value must be computed and compared with the estimation both based on the current value estimator.
A larger difference between the two values, called \textbf{Temporal Difference Error (TD Error)} suggests that more exploration must be done to improve the predictor.
TD Errors and an hyperparamater $\alpha$ are used to compute the probability of each sample.
	\item The part associated to the batch sampling, where \textbf{Importance Sampling Weights} are computed and used to correct the bias introduced by high-priority samples.
\end{itemize}

\subsubsection{Fingerprints}\label{subsubsec:fingerprints}

\subsection{Testing}\label{subsec:testing2}

\section{Curriculum Learning}\label{sec:curriculum-learning}

\begin{quoting}[font=itshape, begintext={"}, endtext={"\citep{bengio_curiculum}}]
The basic idea is to start small, learn easier aspects of the task or easier subtasks, and then gradually increase the difficulty level.\\
\[\dots\]\\
Deep learning methods attempt to learn feature hierarchies.
Features at higher levels are formed by the composition of lower level features.
Automatically learning multiple levels of abstraction may allow a system to induce complex functions mapping the input to the output directly from data, without depending heavily on human-crafted features.
\end{quoting}

Curriculum learning represents an effective bio-inspired strategy to improve learning.
Training with a curriculum accelerate the speed of convergence and may improve the final model performance.
Designing an efficient and effective curriculum is not always easy, incorrect choices may also affect negatively the algorithm.
To overcome this issue many techniques have been proposed.
Curriculum Learning applied to Reinforcement Learning must consider three different practical aspects: how to generate tasks, how to order tasks based on difficulty and how to perform \textbf{Transfer Learning} from task to task~\citep{narvekar2020curriculum}.
Transfer learning has been studied to speed up the learning by providing some initial knowledge rather than starting from zero, it has been successfully applied on policies, models, value functions and more.
There are multiple paradigms and approaches to implement the aforementioned aspects, for example: tasks can be structured into graphs or sequences, can be automatically or manually generated, can be based on the agent behaviour or not (adaptivity), can be classified in one or more types.

\subsection{Implementation}\label{subsec:implementation3}

Task generation is the problem of producing a set of tasks such that knowledge transfer through them is beneficial.
Most of the methods assume the domain can be parameterized using some kind of representation, where different instantiations of these parameters create different tasks.
For instance the Flatland environment provides many parameters to define the degree of freedom of the domain: the grid size, the number of agents, the rail generation, the scheduling and the malfunction stochasticity.
Following this idea~\citep{object-oriented-mdp} proposed a partially automated task generation procedure based on Object-Oriented MDPs.
Considering the Flatland problem as a single-task problem, we proposed two approaches to generate sequences of intermediate tasks in order of difficulty: a fully manual and a semi-automatic.
Similarly to the configuration proposed by the ML-Agents Unity toolkit (\href{https://blogs.unity3d.com/2017/12/08/introducing-ml-agents-v0-2-curriculum-learning-new-environments-and-more}{Introducing ML-Agents Toolkit v0.2}), in the fully manual curriculum we defined a .yml file specifying the parameters of each level of learning.
The semi-automatic allows the user to specify hyperparameters to influence the environment parameters using linear functions and to program short therm and long therm task repeating to deal with "catastrophic forgetting".
Differently from the manual the semi-automatic is evaluated lazily using Python generators.
An important question when designing a curricula is determining the stopping criteria.
Typically training is stopped when performance on the task or set of samples has converged, but another option is to train on each task for a level number of episodes or epochs.
We decided to measure the completion performance and allow levels to be repeated for a finite number of times unless a certain completion threshold is not overcome.
Defining multiple tasks in Flatland is not easy, we hypothesized tasks such as "reaching the goal", "avoiding deadlocks" and "optimizing the path" which can be respectively measured in therms of percentage of completions, number of deadlocks and normalized score.
The problem of generating environment is not easy too because parameters must be tuned to generate environments with feasible solutions and a coherent generation, as instance the number of agents must be lower than the number of cities or the size of the map can influence the maximum number of cities, rails between and within cities.
One alternative approach is to manually describe ranges of coherent parameter values or levels of difficulty that allows an adaptive and automatic process of task sequencing.

\subsection{Testing}\label{subsec:testing3}

\chapter{Conclusions and future works}
% TODO

\newpage
~\nocite{*}
\bibliography{bibliography}

\end{document}
